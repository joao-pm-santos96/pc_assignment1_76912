% This is samplepaper.tex, a sample chapter demonstrating the
% LLNCS macro package for Springer Computer Science proceedings;
% Version 2.20 of 2017/10/04
%
\documentclass[runningheads]{llncs}
%
\usepackage{graphicx}
\usepackage{placeins}
\usepackage[linesnumbered,ruled,vlined]{algorithm2e}
\usepackage[acronym]{glossaries}

\newacronym[]{ir}{IR}{Infra Red}
\newacronym[]{fov}{FOV}{Field of View}

% Used for displaying a sample figure. If possible, figure files should
% be included in EPS format.
%
% If you use the hyperref package, please uncomment the following line
% to display URLs in blue roman font according to Springer's eBook style:
% \renewcommand\UrlFont{\color{blue}\rmfamily}

\begin{document}
%
\title{Markov Localization in the CiberRato simulation environment 
%\thanks{Supported by organization x.}
}
%
%\titlerunning{Abbreviated paper title}
% If the paper title is too long for the running head, you can set
% an abbreviated paper title here
%
\author{João Santos, 76912\inst{1}\orcidID{0000-0003-0393-6456
}}
%
\authorrunning{J. Santos}
% First names are abbreviated in the running head.
% If there are more than two authors, 'et al.' is used.
%
\institute{University of Aveiro, Portugal}
%
\maketitle              % typeset the header of the contribution
%
\begin{abstract}
The abstract should briefly summarize the contents of the paper in
150--250 words. % TODO

\keywords{Peception \and control \and Markov \and localization \and Bayes'.}
\end{abstract}
%
%
% TODO
% - agent or robot?
%
\section{Introduction}
\label{sec:introduction}


\subsection{Environment}
\label{subsec:environment}




\subsection{Assumptions}
\label{subsec:assumptions}

A set of assumptions have been made in order to simplify the implementation. This set is covers not only the specificities of the maps used but also the allowed/expected motions of the robot. They are:

\begin{itemize}
    \item the maps are 7 rows by 14 columns
    \item the unit is the diameter of the robot
    \item cells are squares with side of 2 diameter 
    \item inner walls have a thickness of 0.2 diameter
    \item walls on the border of tha map are 0 thickness
    \item execution ends when the robot hits a wall
    \item motion noise does not exist 
    \item sensor noise is gaussian with mean 0 and variance 0.1
    \item the robot motion is deterministic 
    \item the robot only moves from left to right
    \item the \Gls{ir} sensors detect the closest obstacle measured in the normal direction
    \item the \Gls{ir} sensors have a symmetric \gls{fov} of 60 degrees
\end{itemize}

\subsection{Bayes' Filter}
\label{subsec:bayes_filter}

Given that the map is know to the agent and that the state space is finite we can, in a very direct way, implement the Bayes' filter to, iteratively, compute the the agent localization, or more precisely, to compute the probability of the agent being on any given cell.

Algorithm \ref{alg:bayes_filter} shows the general implementation of the Bayes' filter, as stated on \cite{thrun2005probabilistic}. In it, $bel(x_t)$ represents the belief at time $t$, $u_t$ and $z_t$ are, respectively, the controls inputted and the measurements taken also at time $t$ and $\eta$ is a normalization factor.

\begin{algorithm}[!ht]
    \DontPrintSemicolon
    \KwIn{$bel(x_{t-1}), u_t, z_t$}
    \KwOut{$bel(x_t)$}

    \ForEach{$x_t$}{
        $\overline{bel}(x_t) = \int p(x_t \vert u_t, x_{t-1}) bel(x_{t-1}) dx$
        
        $bel(x_t) = \eta p(z_t \vert x_t) \overline{bel}(x_t)$
    }

\caption{Bayes' filter general algorithm.}
\label{alg:bayes_filter}
\end{algorithm}
\FloatBarrier

Note that for updating the belief at any given instant, only the prior estimate is need, which comes from the Markov assumption.

\subsubsection{Motion Update}

On the scope of this assignment, since the agent is on a finite state space (it either moves or not) and can only move right (motion model) we can say that
%
\begin{equation*} 
    \label{eq1}
    \begin{split}
        \overline{bel}(x_t) &= \int p(x_t \vert u_t, x_{t-1}) bel(x_{t-1}) dx \\
        & = \sum_{x} p(x_t \vert u_1, x) bel(x) \\
        & = p(x_t \vert u_t, x_{t-1}) bel(x_{t-1}) + p(x_t \vert u_1, x_t) bel(x_t)
    \end{split}
\end{equation*}
%
where $p(x_t \vert u_t, x_{t-1})$ and $p(x_t \vert u_1, x_t)$ denote, respectively, the probabilities of the agent has moved right and has stayed stationary, given the control vector $u_t$.

Carried for this assignment, it simply becomes that if the agent has stayed stationary ($u_t = [0, 0]$) then $p(x_t \vert u_t, x_{t-1}) = 0$ and $p(x_t \vert u_1, x_t) = 1$, and vice versa if the agent has moved ($u_t = [1, 0]$).

\subsubsection{Measuremnet Integration}

It is in this step were the data gathered from all sensors is fused. Given that any given measure from each of the sensors follows the $\sim\mathcal{N}(true\_value, 0.1)$ model, we can compute $bel(x_t) = \eta p(z_t \vert x_t) \overline{bel}(x_t)$ from algorithm \ref{alg:bayes_filter} with algorithm \ref{alg:bayes_sense}, where $gt(x)$ represents the ground truth values (measurements expected to be gathered in each cell) for a given sensor in cell $x$. This ground truth was computed a priori since the map is known to the agent.

\begin{algorithm}[!ht]
    \DontPrintSemicolon
    \KwIn{$bel(x_{t-1}), z_t, gt$}
    \KwOut{$bel(x_t)$}

    \ForEach{$x_t$}{
        $product \gets 1$ 

        $\eta \gets 0$

        \ForEach{sensor}{
            $ product = product \times \mathcal{N}(gt(x) - z_t, 0.1) $ 
        }

        $\overline{bel}(x_t) = product \times bel(x_{t-1})$

        $\eta = \eta + \overline{bel}(x_t)$
    }

    \ForEach{$x_t$}{
        $bel(x_{t}) = \eta^{-1} \overline{bel}(x_t)$
    }

\caption{Measurement integration algorithm.}
\label{alg:bayes_sense}
\end{algorithm}
\FloatBarrier



\section{Implementation}
\label{sec:implementation}

As stated on section \ref{subsec:assumptions}, the robot only moves in a straight line, from left to right. This poses several advantages, the first of which is that the walls are always perpendicular or parallel to the robot's motion direction. To take advantage of this, the author choose to set the robot's four \gls{ir} distance sensors perpendicular to the cell's border, i. e., to 0, 90, -90 and 180 degrees measured from the robot's travelling direction.










\subsection{Debug Tools}
\label{subsec:debug_tools}





\section{Results}
\label{sec:results}






























\iffalse
\section{First Section}
\subsection{A Subsection Sample}
Please note that the first paragraph of a section or subsection is
not indented. The first paragraph that follows a table, figure,
equation etc. does not need an indent, either.

Subsequent paragraphs, however, are indented.

\subsubsection{Sample Heading (Third Level)} Only two levels of
headings should be numbered. Lower level headings remain unnumbered;
they are formatted as run-in headings.

\paragraph{Sample Heading (Fourth Level)}
The contribution should contain no more than four levels of
headings. Table~\ref{tab1} gives a summary of all heading levels.

\begin{table}
\caption{Table captions should be placed above the
tables.}\label{tab1}
\begin{tabular}{|l|l|l|}
\hline
Heading level &  Example & Font size and style\\
\hline
Title (centered) &  {\Large\bfseries Lecture Notes} & 14 point, bold\\
1st-level heading &  {\large\bfseries 1 Introduction} & 12 point, bold\\
2nd-level heading & {\bfseries 2.1 Printing Area} & 10 point, bold\\
3rd-level heading & {\bfseries Run-in Heading in Bold.} Text follows & 10 point, bold\\
4th-level heading & {\itshape Lowest Level Heading.} Text follows & 10 point, italic\\
\hline
\end{tabular}
\end{table}


\noindent Displayed equations are centered and set on a separate
line.
\begin{equation}
x + y = z
\end{equation}
Please try to avoid rasterized images for line-art diagrams and
schemas. Whenever possible, use vector graphics instead (see
Fig.~\ref{fig1}).

\begin{figure}
\includegraphics[width=\textwidth]{fig1.eps}
\caption{A figure caption is always placed below the illustration.
Please note that short captions are centered, while long ones are
justified by the macro package automatically.} \label{fig1}
\end{figure}

\begin{theorem}
This is a sample theorem. The run-in heading is set in bold, while
the following text appears in italics. Definitions, lemmas,
propositions, and corollaries are styled the same way.
\end{theorem}
%
% the environments 'definition', 'lemma', 'proposition', 'corollary',
% 'remark', and 'example' are defined in the LLNCS documentclass as well.
%
\begin{proof}
Proofs, examples, and remarks have the initial word in italics,
while the following text appears in normal font.
\end{proof}
For citations of references, we prefer the use of square brackets
and consecutive numbers. Citations using labels or the author/year
convention are also acceptable. The following bibliography provides
a sample reference list with entries for journal
articles~\cite{ref_article1}, an LNCS chapter~\cite{ref_lncs1}, a
book~\cite{ref_book1}, proceedings without editors~\cite{ref_proc1},
and a homepage~\cite{ref_url1}. Multiple citations are grouped
\cite{ref_article1,ref_lncs1,ref_book1},
\cite{ref_article1,ref_book1,ref_proc1,ref_url1}.
\fi


%
% ---- Bibliography ----
%
% BibTeX users should specify bibliography style 'splncs04'.
% References will then be sorted and formatted in the correct style.
%
\bibliographystyle{splncs04}
\bibliography{bib}
%
% \begin{thebibliography}{8}
% % \bibitem{ref_article1}
% % Author, F.: Article title. Journal \textbf{2}(5), 99--110 (2016)

% % \bibitem{ref_lncs1}
% % Author, F., Author, S.: Title of a proceedings paper. In: Editor,
% % F., Editor, S. (eds.) CONFERENCE 2016, LNCS, vol. 9999, pp. 1--13.
% % Springer, Heidelberg (2016). \doi{10.10007/1234567890}

% % \bibitem{ref_book1}
% % Author, F., Author, S., Author, T.: Book title. 2nd edn. Publisher,
% % Location (1999)

% % \bibitem{ref_proc1}
% % Author, A.-B.: Contribution title. In: 9th International Proceedings
% % on Proceedings, pp. 1--2. Publisher, Location (2010)

% % \bibitem{ref_url1}
% % LNCS Homepage, \url{http://www.springer.com/lncs}. Last accessed 4
% % Oct 2017

% \end{thebibliography}
\end{document}
